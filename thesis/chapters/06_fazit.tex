\subsection{Zusammenfassung und Erkenntnisse}
Die Experimente bestätigen, dass eine zuverlässige Vorhersage von Rennergebnissen möglich ist. Bei der Klassifikation der Punkteplatzierungen lieferte das TabPFN-Modell mit einem F1-Score von 0,855 die besten Ergebnisse, während in der Regression der Random Forest (MAE 21,8\,s) am stärksten abschnitt. Entscheidend für den Erfolg waren dabei die Nutzung des Zeitabstands zum Sieger als robuste Zielvariable sowie eine saubere Datentrennung nach Jahren. Die Genauigkeit wird derzeit jedoch noch durch die fehlende Modellierung von Ausfällen (DNFs) und chaotischen Rennverläufen eingeschränkt. Die Ausfallquote wurde mit dem Skript \texttt{calc\_retirement\_rate.py} berechnet und zeigte in den Jahren von 2015-2025 eine durchschnittliche Rate von etwas 16,50\%. Das könnte eine mögliche Limitation sein, welche es nicht erlaubt noch höhere Genauigkeiten zu erzielen.

\subsection{Ausblick}
Künftige Arbeiten sollten zusätzliche Kontextdaten wie Safety-Car-Phasen und Boxenstopps integrieren, um die Realität besser abzubilden. Darüber hinaus wäre es interessant, mit Live-Daten bis zu einem definierten Zeitpunkt im Rennen zu arbeiten, etwa bis zur Rennhälfte oder bis zum letzten Boxenstopp. Dadurch könnten Zwischenstände, aktuelle Pace und strategische Entscheidungen direkt berücksichtigt werden, was die Vorhersagegenauigkeit insbesondere bei späten Rennereignissen erhöhen dürfte.
