\subsection{Hintergrund und Problemstellung}
Die Formel 1 ist ein stark datengetriebener Sport, weshalb sich die Leistung von Fahrern und Teams gut analysieren lässt. Die zentrale Frage dieser Arbeit ist, ob man aus den vor einem Rennen verfügbaren Informationen zuverlässige Ergebnisse vorhersagen kann. Das Problem wird dabei zweigeteilt: Zunächst wird klassifiziert, ob ein Fahrer überhaupt Punkte erzielt. Darauf aufbauend wird der zeitliche Rückstand zum Sieger modelliert, da dieser Wert eine stabilere Zielvariable darstellt als die durch äußere Einflüsse oft verfälschte absolute Rennzeit.

\subsection{Zielsetzung und Ressourcen}
Das Ziel dieser Arbeit ist der Aufbau eines reproduzierbaren Datensatzes sowie die Entwicklung und der Vergleich geeigneter Machine-Learning-Modelle. Unter Einbeziehung verschiedener Merkmale, wie etwa der Startposition oder des Streckentyps, werden die Modelle trainiert und anschließend evaluiert. Dies dient als Basis, um perspektivisch auch Podiumsplatzierungen vorhersagen zu können. Der gesamte Code, der in diesem Projekt verwendet wurde, ist unter \href{https://github.com/mcfiet/dl-project}{https://github.com/mcfiet/dl-project} zu finden.