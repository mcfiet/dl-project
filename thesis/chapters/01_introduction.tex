\subsection{Code und Daten}
Der Code der in diesem Projekt verwendet wurde ist unter \href{https://github.com/mcfiet/dl-project}{https://github.com/mcfiet/dl-project} zu finden.

\subsection{Motivation}
Die Formel 1 ist ein datengetriebener Sport, in dem Performance von vielen Faktoren
abhängt: Fahrer, Team, Strecke, Startposition und Rennverlauf. Durch die
Verfügbarkeit strukturierter Telemetrie- und Ergebnisdaten lässt sich genau bestimmen, wie stark diese Einflüsse sind. Vorhersagen sind für Fans,
Teams und Medien interessant, weil sie Vergleiche über Saisons hinweg
ermöglichen und Entscheidungen (z.B. Strategieeinschätzungen) stützen
können.

\subsection{Problemstellung}
Die zentrale Frage dieser Arbeit ist, ob sich aus vor dem Rennen verfügbaren
Informationen eine zuverlässige Vorhersage über Rennergebnisse ableiten lässt.
Konkret wird ein Klassifikationsproblem betrachtet: Für jeden Fahrer soll
vorhergesagt werden, ob er in einem Grand Prix Punkte erzielt. 

Darauf aufbauend wird ein Regressionsproblem formuliert, das den Abstand zum
Sieger (\texttt{gap\_to\_winner}) in Sekunden modelliert. Absolute Rennzeiten
zeigen sich als zu verrauscht (DNFs, rote Flaggen, Safety Cars); der Gap dient
als stabileres, saisonübergreifendes Target und erlaubt perspektivisch eine
Positions- bzw. Podiumsprognose. Die Herausforderung liegt in der Mischung aus
kategorischen und numerischen Merkmalen, zeitlicher Dynamik sowie in der
begrenzten Anzahl an Beispielen pro Saison.

\subsection{Ziel dieser Arbeit}
Ziel ist der Aufbau eines reproduzierbaren Datensatzes aus FastF1-Daten sowie
die Entwicklung und der Vergleich geeigneter Machine-Learning-Modelle für die
Punktevorhersage. Dazu werden Merkmale wie Startposition, Qualifying-Performance,
Saisonschnitt der Punkte und Streckentypen abgeleitet. Die Modelle werden auf
Trainings-, Validierungs- und Testdaten evaluiert und mit geeigneten Metriken
(u.a. F1-Score und Balanced Accuracy) verglichen.

Ähnlich wie bei der Klassifikation, wird auch bei der Regression ein Builder entwickelt, der konsistente
Rennzeiten rekonstruiert, Gaps ableitet und robuste Fehlermaße (MAE/RMSE,
SMAPE) verwendet. Ziel ist eine belastbare Basis, um anschließend Podiums- oder
Siegerprognosen abzuleiten. Abschließend werden die Ergebnisse interpretiert
und Limitationen sowie mögliche Erweiterungen diskutiert.
