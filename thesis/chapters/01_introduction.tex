\subsection{Code und Daten}
Der Code der in diesem Projekt verwendet wurde ist unter \href{https://github.com/mcfiet/dl-project}{https://github.com/mcfiet/dl-project} zu finden.

\subsection{Motivation}
Die Formel 1 ist ein datengetriebener Sport, in dem Performance von vielen Faktoren
abhängt: Fahrer, Team, Strecke, Startposition und Rennverlauf. Durch die
Verfügbarkeit strukturierter Telemetrie- und Ergebnisdaten ist es möglich,
solche Einflüsse quantitativ zu modellieren. Vorhersagen sind für Fans,
Teams und Medien interessant, weil sie Vergleiche über Saisons hinweg
ermöglichen und Entscheidungen (z.B. Strategieeinschätzungen) stützen
können.

\subsection{Problemstellung}
Die zentrale Frage dieser Arbeit ist, ob sich aus vor dem Rennen verfügbaren
Informationen eine zuverlässige Vorhersage über Rennergebnisse ableiten lässt.
Konkret wird ein Klassifikationsproblem betrachtet: Für jeden Fahrer soll
vorhergesagt werden, ob er in einem Grand Prix Punkte erzielt. 

Im nächsten Schritt soll ein Regressionsproblem betrachtet werden: So soll dann über die Rennzeit die Position in einem Grand Prix vorhergesagt werden. Damit soll später ein Podium, dann der Sieger usw. vorhergesagt werden. Die Herausforderung liegt in der Heterogenität der Daten (kategorische und numerische Merkmale), der saisonalen Dynamik sowie in der begrenzten Menge an Beispielen pro Saison.

\subsection{Ziel dieser Arbeit}
Ziel ist der Aufbau eines reproduzierbaren Datensatzes aus FastF1-Daten sowie
die Entwicklung und der Vergleich geeigneter Machine-Learning-Modelle für die
Punktevorhersage. Dazu werden Merkmale wie Startposition, Qualifying-Performance,
Saisonschnitt der Punkte und Streckentypen abgeleitet. Die Modelle werden auf
Trainings-, Validierungs- und Testdaten evaluiert und mit geeigneten Metriken
(u.a. F1-Score und Balanced Accuracy) verglichen. Abschließend werden die
Ergebnisse interpretiert und Limitationen sowie mögliche Erweiterungen
diskutiert.

