In der Forschung gibt es bisher wenige Arbeiten, die sich mit der Vorhersage von Rennergebnissen in der Formel 1 beschäftigen. Es gibt einige private Projekte und Blogs, die sich mit diesem Thema auseinandersetzen \parencite[vgl.][]{mehta_f1_2019}, jedoch fehlt es an wissenschaftlichen Veröffentlichungen, die systematisch verschiedene Ansätze vergleichen und evaluieren. Einige Lösungen nutzen jedoch vergleichbare Datensätze und Merkmale, wie sie in dieser Arbeit verwendet werden. Oft wurde das Problem mit einem Random Forest Classifier  \parencite[vgl.][]{weng_formula1_2020} oder Gradient Boosting Modellen \parencite[vgl.][]{antaya_2025_f1_predictions} angegangen, da diese Modelle gut mit tabellarischen Daten umgehen können und weniger anfällig für Overfitting sind.